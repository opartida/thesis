\documentclass[11pt,amsmath,amssymb]{article}

%\documentclass[twocolumn,amssymb,secnumarabic,preprintnumbers, %amsmath]{revtex4}
%\pdfoutput=1
%\documentclass[aps,10pt,twocolumn,showpacs,amsmath,amssymb,nofootinbib]{revtex4-1}

\usepackage{graphicx}
\usepackage{color}
\usepackage{soul}


%\pdfoutput=1 % if your are submitting a pdflatex (i.e. if you have
             % images in pdf, png or jpg format)

%\usepackage{jheppub} % for details on the use of the package, please
                     % see the JHEP-author-manual

\usepackage{longtable}
%\usepackage{widetable}
%\topmargin -1.5cm

\usepackage{graphicx}
%\usepackage{amssymb}
\usepackage{booktabs}
%\usepackage{color}
\usepackage{placeins}
\usepackage{multirow}
\usepackage{url}
%\usepackage{hyperref}

% \extracolsep{\fill}

\newcommand{\ra}[1]{\renewcommand{\arraystretch}{#1}}
\newcommand{\rb}[1]{\renewcommand{\tabcolsep}{#1}}

\usepackage{amsmath}
\usepackage{amssymb}

\newcommand{\C}[1]{{\cal C}_{#1}}
\newcommand{\B}{{\cal B}}

\newcommand{\ac}[1]{{\color{blue} #1}}

\newcommand{\ord}{\mathcal{O}}
\newcommand{\nn}{\noindent}
\newcommand{\nb}{\nonumber}
\newcommand{\arccot}{{\rm arccot}}
\newcommand{\sss}{\scriptscriptstyle}
\newcommand{\dis}{\displaystyle}
\newcommand{\eq}[1]{\begin{equation} #1 \end{equation}}
\newcommand{\eqa}[1]{\begin{eqnarray} #1 \end{eqnarray}}
\newcommand{\Cc}[1]{{\cal C}_{#1}}
\newcommand{\Cp}[1]{{\cal C}_{#1}'}
\newcommand{\delC}[1]{\delta {\cal C}_{#1}}
\newcommand{\dC}[1]{{\cal C}_{#1}^{\rm NP}}
\newcommand{\dCp}[1]{{\cal C}_{#1^\prime}^{\rm NP}}
\newcommand{\av}[1]{\langle #1 \rangle}
\newcommand{\afb}{A_{\rm FB}}
\newcommand{\red}[1]{{\color{red} #1}}
\newcommand{\qm}[1]{{\color{blue} #1}}
\newcommand{\jv}[1]{{\red {#1}}}

\newcommand{\sect}[1]{\section{\hspace{-0.3cm} #1}}

\begin{document}

\title{Proposal for B-meson decay analysis}
\author{Pere Masjuan}
\date{\today}
\maketitle

\section{State-of-the-art}

Current analysis of anomalies in flavour physics are based on a linear regression of a $\chi^2$ function.
After taking into account correlation between theoretical predictions and experimental observables,
a $\chi^2$ function is built and minimized.

Currently, 180 observables are considered from different collaborations~\cite{Aaij:2019wad,BelleRK,Aaij:2014pli,Abdesselam:2019wac,Aaboud:2018mst,Wehle:2016yoi,Abdesselam:2016llu}. Observable is referred to a measurement of either an angular
observable, a branching ratio, or a ratio, in a particular energy bin. For $B \to K \mu\mu$, the available 
energy of the $\mu\mu$ invariant mass (the so called $q^2$ variable), runs from $0$ to $(m_B - m_K)^2 \sim 22$GeV$^2$
while for the $B \to K* \mu\mu$ runs from $0$ to $(m_B - m_{K*})^2 \sim 19$GeV$^2$.
In this exercise, charged and neutral modes of the $B$ meson are considered. This brings, $B^{0,\pm} \to K^{0,\pm}$ which
are equivalent up to isospin corrections, thus important to include all of them for enlarging the data set. We also count
with two different leptonic final states, $\mu\mu$ and $e+e-$. Rations of the same observable but with different lepton
at the final state, the so-called $R$ ratios, are also included and play an important psychological goal since they are a clear
indication of the Lepton Flavor Universality Violation - LFUV - (this is, that the Z and W bosons of the SM interact differently with electrons
and muons, something which is not expected from the theory)~\cite{Vicente:2020usa}.
On the contrary, deviations between theory and experiment for specific lepton, either muons or electrons, cannot distinguish
whether the anomaly is due to LFUV phenomenon or new physics come universally for muons and electrons. Only the combination
of lepton-flavor dependent and rations of observables with different leptons in the final state may help. But actually, there is no
clean way to disentangle such situation and current attempts consider that the most appealing solution to the puzzle is indeed
having two different new physics scenarios, two different new-physics particles, one affecting all lepton flavours and a second one
affecting exclusively the muonic modes~\cite{Alguero:2018nvb}.

For a given decay process, for example, the branching ratio $B^0 \to K^{*0} \mu\mu$ we will have five or six energy bins,
with each bin having 1 or 2 GeV$^2$ sizes, thus providing 5 of 6 so-called observables into the fit.

For each measured observable, we have a theory prediction based on the Standard Model of Particle Physics (SM).
With 180 observables, the $\chi^2$ value of the SM reaches 225 points~\cite{Alguero:2019ptt}, which corresponds
to a p-value of 1.4$\%$. This indicates the SM to be very far to explain experimental measurements.

The strategy then has been to include on top of the SM, new operators in the effective Hamiltonian to be able
to account for such experimental discrepancies.

Standard strategy, the simplest one, considers one-single operator at a time, the so-called 1D fits, starting with ${\cal C}_7$,
${\cal C}_9$ and ${\cal C}_{10}$. For example, in Table~\ref{tab:results1D}, some examples can be found from Ref.~\cite{Alguero:2019ptt}.
The first one, a fit with $\Cc{9\mu}^{\rm NP}$ returns $\Cc{9\mu}^{\rm NP} = -0.98$. The $\Cc{9\mu}^{\rm SM} \sim 4$, so 
the NP correction to the SM to explain these anomalies are around $25\%$, pretty large. Pull$_{\rm SM}$ reflects how difficult
would be for that model (defined as SM + $\Cc{9\mu}^{\rm NP}$ ) would be able to explain the SM. The p-value is then a criterion
to decide whether this fit is rejected or not. Since p-values are very large, the fit is very good compared to the SM p-value of 1.4$\%$.
 
 \begin{table*}[h!] \scriptsize
\begin{tabular}{c||c|c|c|c||c|c|c|c} %\scriptsize
 & \multicolumn{4}{c||}{All} &  \multicolumn{4}{c}{LFUV}\\
\hline
1D Hyp.   & Best fit& 1 $\sigma$/2 $\sigma$   & Pull$_{\rm SM}$ & p-value & Best fit & 1 $\sigma$/ 2 $\sigma$  & Pull$_{\rm SM}$ & p-value\\
\hline\hline
\multirow{2}{*}{$\Cc{9\mu}^{\rm NP}$}    & \multirow{2}{*}{-0.98} &    $[-1.15,-0.81]$ &    \multirow{2}{*}{5.6}   & \multirow{2}{*}{65.4\,\%}
&   \multirow{2}{*}{-0.89}   &$[-1.23,-0.59]$&   \multirow{2}{*}{3.3}  & \multirow{2}{*}{52.2\,\%}  \\
 &  & $[-1.31,-0.64]$ &  & &  &  $[-1.60,-0.32]$ & \\
 \multirow{2}{*}{$\Cc{9\mu}^{\rm NP}=-\Cc{10\mu}^{\rm NP}$}    &   \multirow{2}{*}{-0.46} &    $[-0.56,-0.37]$ &   \multirow{2}{*}{5.2}  & \multirow{2}{*}{55.6\,\%}
 &  \multirow{2}{*}{-0.40}   &   $[-0.53,-0.29]$ & \multirow{2}{*}{4.0}   & \multirow{2}{*}{74.0\,\%}  \\
 &  & $[-0.66,-0.28]$ &  & & & $[-0.63,-0.18]$  &    \\
 \multirow{2}{*}{$\Cc{9\mu}^{\rm NP}=-\Cc{9'\mu}$}     & \multirow{2}{*}{-0.99} &    $[-1.15,-0.82]$   &  \multirow{2}{*}{5.5}  & \multirow{2}{*}{62.9\,\%}
 &  \multirow{2}{*}{-1.61}   &    $[-2.13,-0.96]$  & \multirow{2}{*}{3.0} & \multirow{2}{*}{42.5\,\%} \\
 &  & $[-1.31,-0.64]$ &  & & & $[-2.54,-0.41]$ &    \\
\hline
 \multirow{2}{*}{$\Cc{9\mu}^{\rm NP}=-3 \Cc{9e}^{\rm NP}$} & \multirow{2}{*}{-0.87}  & $[-1.03,-0.71]$ & \multirow{2}{*}{5.5}  & \multirow{2}{*}{61.9\,\%}
  &   \multirow{2}{*}{-0.66} &    $[-0.90,-0.44]$ & \multirow{2}{*}{3.3}  & \multirow{2}{*}{52.2\,\%}
\\
 & & $[-1.19,-0.55]$ &  & & & $[-1.17,-0.24]$     & \\
\end{tabular} \scriptsize
\caption{Most prominent 1D patterns of NP in $b\to s\mu^+\mu^-$. Pull$_{\rm SM}$ is quoted in units of standard deviation. The $p$-value of the SM hypothesis is 11.0\% for the fit ``All" and 8.0\% for the fit LFUV.} \scriptsize
\label{tab:results1D}
\end{table*}


\section{Aim of this project}

Use a neural network to determine what combination of new operators renders the best possible fit to experimental data.

The intuition can be obtained by looking at the tables of Ref.~\cite{Alguero:2019ptt}, one of them reproduced here.
Forget by now about coefficients with a prime such as $\Cc{9'}$ or $\Cc{10'}$. The difference between $\Cc{9}^{\rm V}$ and $\Cc{9}^{\rm U}$
is that $V$ refers to muon channel exclusively while $U$ means both muons and electrons.
In the long table, different ad-hoc combinations of wilson coefficients are considered. There are, of course, other options. The question is what
is the particular combnation that renders the largest agreement with data without using all coeficients free. This is, instead of using, for example
4 parameters $\Cc{9}^{\rm V}$, $\Cc{9}^{\rm U}$ and $\Cc{10}^{\rm V}$, $\Cc{10}^{\rm U}$, what combination of two parameters (made from the
combination of these four, for example $\Cc{9}^{\rm V} = \pm \Cc{9}^{\rm U}$, $\Cc{9}^{\rm V} = \pm \Cc{10}^{\rm V}$) yields the best agreement.

The goal is then try to look for these combination via exploring the many options using a neural network. The input shall be a fit with one single
parameter at the begining, and explore different $\chi^2$ values using different input coefficients for each observable and then let the neural
network decide the best strategy. Combining 2D fits, with two different combinations of coeficients, arbitrary cobination, also provides information
for the NN to learn, and it is proved that this is most eficient.

\begin{table*}[h!] \small \begin{center}
\begin{tabular}{lc||c|c|c|c|c}
\multicolumn{2}{c||}{Scenario} & Best-fit point & 1 $\sigma$ & 2 $\sigma$ & Pull$_{\rm SM}$ & p-value \\
\hline\hline
\multirow{ 3}{*}{Scenario 5} &$\Cc{9\mu}^{\rm V}$ & $-0.54$ & $[-1.06,-0.06]$ & $[-1.68,+0.39]$ &
\multirow{ 3}{*}{6.0} & \multirow{ 3}{*}{39.4\,\%} \\
&$\Cc{10\mu}^{\rm V}$ & $+0.58$ & $[+0.13,+0.97]$ & $[-0.48,+1.33]$ & \\
&$\Cc{9}^{\rm U}=\Cc{10}^{\rm U}$ & $-0.43$ & $[-0.85,+0.05]$ & $[-1.23,+0.67]$ &\\
\hline
\multirow{ 2}{*}{Scenario 6}&$\Cc{9\mu}^{\rm V}=-\Cc{10\mu}^{\rm V}$ & $-0.56$ & $[-0.65,-0.47]$ & $[-0.75,-0.38]$ &
\multirow{ 2}{*}{6.2} & \multirow{ 2}{*}{41.4\,\%} \\
&$\Cc{9}^{\rm U}=\Cc{10}^{\rm U}$ & $-0.41$ & $[-0.53,-0.29]$ & $[-0.64,-0.16]$ &\\
\hline
\multirow{ 2}{*}{Scenario 7}&$\Cc{9\mu}^{\rm V}$ & $-0.84$ & $[-1.15,-0.54]$ & $[-1.48,-0.26]$ &
\multirow{ 2}{*}{6.0} & \multirow{ 2}{*}{36.5\,\% }  \\
&$\Cc{9}^{\rm U}$ & $-0.25$ & $[-0.59,+0.10]$ & $[-0.92,+0.47]$  &\\
\hline
\multirow{ 2}{*}{Scenario 8}&$\Cc{9\mu}^{\rm V}=-\Cc{10\mu}^{\rm V}$ & $-0.34$ & $[-0.44,-0.25]$ & $[-0.54,-0.16]$ &
\multirow{ 2}{*}{6.5} & \multirow{ 2}{*}{48.4\,\%} \\
&$\Cc{9}^{\rm U}$ & $-0.80$ & $[-0.98,-0.60]$ & $[-1.16,-0.39]$ &\\
\hline\hline
\multirow{ 2}{*}{Scenario 9}&$\Cc{9\mu}^{\rm V}=-\Cc{10\mu}^{\rm V}$ & $-0.66$ & $[-0.79,-0.52]$ & $[-0.93,-0.40]$ &
\multirow{ 2}{*}{5.7} & \multirow{ 2}{*}{28.4\,\%} \\
&$\Cc{10}^{\rm U}$ & $-0.40$ & $[-0.63,-0.17]$ & $[-0.86,+0.07]$ &\\
\hline
\multirow{ 2}{*}{Scenario 10}&$\Cc{9\mu}^{\rm V}$ & $-1.03$ & $[-1.18,-0.87]$ & $[-1.33,-0.71]$ &
\multirow{ 2}{*}{6.2} & \multirow{ 2}{*}{41.5\,\%} \\
&$\Cc{10}^{\rm U}$ & $+0.28$ & $[+0.12,+0.45]$ & $[-0.04,+0.62]$ &\\
\hline
\multirow{ 2}{*}{Scenario 11}&$\Cc{9\mu}^{\rm V}$ & $-1.11$ & $[-1.26,-0.95]$ & $[-1.40,-0.78]$ &
\multirow{ 2}{*}{6.3} & \multirow{ 2}{*}{43.9\,\%} \\
&$\Cc{10'}^{\rm U}$ & $-0.29$ & $[-0.44,-0.15]$ & $[-0.58,-0.01]$ &\\
\hline
\multirow{ 2}{*}{Scenario 12}&$\Cc{9'\mu}^{\rm V}$ & $-0.06$ & $[-0.21,+0.10]$ & $[-0.37,+0.26]$ &
\multirow{ 2}{*}{2.1} & \multirow{ 2}{*}{2.2\,\%} \\
&$\Cc{10}^{\rm U}$ & $+0.44$ & $[+0.26,+0.62]$ & $[+0.09,+0.81]$ &\\
\hline
\multirow{ 4}{*}{Scenario 13}&$\Cc{9\mu}^{\rm V}$ & $-1.16$ & $[-1.31,-1.00]$ & $[-1.46,-0.83]$ &
\multirow{ 4}{*}{6.2} & \multirow{ 4}{*}{49.2\,\%} \\
&$\Cc{9'\mu}^{\rm V}$ & $+0.56$ & $[+0.27,+0.83]$ & $[-0.02,+1.10]$ &\\
&$\Cc{10}^{\rm U}$ & $+0.28$ & $[+0.08,+0.49]$ & $[-0.11,+0.70]$ &\\
&$\Cc{10'}^{\rm U}$ & $+0.01$ & $[-0.19,+0.22]$ & $[-0.40,+0.42]$ &\\
\end{tabular}
\caption{Most prominent patterns for LFU and LFUV NP contributions from Fit ``All''.
Scenarios 5 to 8 were introduced in Ref.~\cite{Alguero:2018nvb}.  Scenarios 9 (motivated by 2HDMs~\cite{Crivellin:2019dun}) and 10 to 13  (motivated by $Z^\prime$ models with vector-like quarks~\cite{Bobeth:2016llm}) are newly introduced in the main text.}\label{Fit3DbisApp} \end{center}
\end{table*}


\begin{equation}
\chi^2 = \sum_i \frac{ ( y^{\rm exp}_i - {\rm Model}_i )^2}{ \sigma {\rm Exp}^2 + \sigma {\rm Model}^2}\, ,
\end{equation}
\noindent
where Model refers to the theoretical calculation. Following the same idea, \textit{pull} is
\begin{equation}
{\rm Pull}_i = \frac{ ( y^{\rm exp}_i - {\rm Model}_i )}{ \sqrt{\sigma {\rm Exp}^2 + \sigma {\rm Model}^2} }\
\end{equation}





\section{Trining exercise}

Before starting with full fits with 180 observables, we shall explore the mechanics with a just few of them that can be easily parameterized 
as a polynomial. The example considers the $\langle P'_5 \rangle$ observable, the most famous one, in different bins.
I run my codes to generate a polynomial description for this observable. In the table, experimental results and SM predictions for the observable
are collected.

\begin{table}[ht]
\caption{$\langle P'_5 \rangle$ bins, SM predictions, experimental results from LHCb collaboration, and Pulls between SM and experiment.}
\begin{center}
\begin{tabular}{|c|c|c|c|}
\hline
bin & SM & experiment & Pull\\
\hline
$( 0.1,0.98 )$ &   0.67 $\pm 0.14$ & 0.52 $\pm 0.10$ & +0.9\\
$( 1.1, 2.5 )$ &   0.20 $\pm 0.12$ & 0.37 $\pm 0.12$ & -1.0\\
$( 2.5,4 )$ &   -0.49 $\pm 0.12$ & -0.15 $\pm 0.15$ & -1.8\\
$( 4,6 )$ &   -0.83 $\pm 0.08$ & -0.44 $\pm 0.12$ & -2.7\\
$( 6,8 )$ &   -0.94 $\pm 0.08$ & -0.58 $\pm 0.09$ & -2.9\\
$( 15,19 )$ &   -0.57 $\pm 0.05$ & -0.67 $\pm 0.06$ & +1.2\\
\hline
\end{tabular}
\end{center}
\label{p5presults}
\end{table}%

We will use the following parameterization for the observables.
\begin{eqnarray}\nonumber
\langle P'_5 \rangle_{[0.1,0.98]} & = & 0.674 - 0.060 C_{10} - 0.012 C_{10}^2 - 0.104 C_{9} - 0.010 C_{9}C_{10} + 0.005 C_{9}^2 \\ \nonumber
\langle P'_5 \rangle_{[1.1,2.5]} & = & 0.196- 0.001 C_{10} - 0.004 C_{10}^2 - 0.210 C_{9} + 0.001 C_{9}C_{10} + 0.002 C_{9}^2 \\ \nonumber
\langle P'_5 \rangle_{[2.5,4]} & = & -0.491- 0.042 C_{10} + 0.001 C_{10}^2 - 0.285 C_{9} - 0.020 C_{9}C_{10} + 0.033 C_{9}^2 \\ \nonumber
\langle P'_5 \rangle_{[4,6]} & = & -0.826 - 0.066 C_{10} + 0.004 C_{10}^2 - 0.207 C_{9} + 0.011 C_{9}C_{10} + 0.058 C_{9}^2 \\ \nonumber
\langle P'_5 \rangle_{[6,8]} & = & -0.937 - 0.060 C_{10} + 0.009 C_{10}^2 - 0.131 C_{9} + 0.036 C_{9}C_{10} + 0.055 C_{9}^2 \\ \nonumber
\langle P'_5 \rangle_{[15,19]} & = & -0.572 - 0.021 C_{10} + 0.010 C_{10}^2 - 0.040 C_{9} + 0.029 C_{9}C_{10} + 0.025 C_{9}^2 \\ \nonumber
\end{eqnarray}
\noindent
Notice that for $\Cc{9} = \Cc{10} = 0$, we recover the SM prediction.  
  
Imagine, you solve   $\langle P'_5 \rangle_{[0.1,0.98]}  = 0.52 \pm 0.10$ but using only $\Cc{9}$, you get $\Cc{9} = 1.58$. However, if you take
$\langle P'_5 \rangle_{[4,6]}  = -0.83 \pm 0.08$, you get $\Cc{9} = -1.32$ which has nothing to do.
A combined fit to all these observables returns $\Cc{9} = -1.06$ with $\chi^2_{\rm min} = 9.2$, which is not really a good fit. Moreover, if we neglect
the last bin, the fit returns $\Cc{9} = -1.24$ with $\chi^2_{\rm min} = 4.0$, much better. The question is then, what observable weights more in the fit
and how can be put into agreement the different tensions.

\begin{figure}[htbp]
\begin{center}
%\includegraphics[width=15cm]{NN.pdf}
%\caption{Simplistic view of a neural network.}
%\label{NN}
\end{center}
\end{figure}


Figure~\ref{NN} represents the final philosophy. Each observable (yellow circle) has an experimental value ($x_i$) 
depending on the bin you are looking at. For $\langle P'_5 \rangle$ we have 6 different bins, so we will have 6 different $x_i$.
Each of it can include $\Cc{9}$, $\Cc{10}$, with $V,U$, and all these options are represented with $y_i$. The combination of
observable in a particular bin, with a fit with a ${\cal C}$ coefficient, returns a $\chi^2$ value.

The NN should learn how to reach the best $\chi^2$ by exploring all the possible combinations of ${\cal C}$ wilson coefficients
with the experimental values and learn what is the optimal weights of the observables.

Two different strategies to be compared:
\begin{itemize}
\item look for the minimum $\chi^2$
\item look for the solution with closer $\chi^2/dof = 1$ possible.
\end{itemize}


  
\begin{thebibliography}{99}

%\cite{Alguero:2019ptt}
\bibitem{Alguero:2019ptt}
M.~Alguer\'o, B.~Capdevila, A.~Crivellin, S.~Descotes-Genon, P.~Masjuan, J.~Matias, M.~Novoa and J.~Virto,
%``Emerging patterns of New Physics with and without Lepton Flavour Universal contributions,''
Eur. Phys. J. C \textbf{79} (2019) no.8, 714
%doi:10.1140/epjc/s10052-019-7216-3
[arXiv:1903.09578 [hep-ph]].
%82 citations counted in INSPIRE as of 17 Apr 2020

\bibitem{Alguero:2018nvb}
  M.~Alguer\'o, B.~Capdevila, S.~Descotes-Genon, P.~Masjuan and J.~Matias,
  %``Are we overlooking lepton flavour universal new physics in $b\to s\ell\ell$ ?,''
  Phys.\ Rev.\ D {\bf 99} (2019) no.7,  075017
  %doi:10.1103/PhysRevD.99.075017
  [arXiv:1809.08447 [hep-ph]].
  %%CITATION = doi:10.1103/PhysRevD.99.075017;%%
  
%\cite{Vicente:2020usa}
\bibitem{Vicente:2020usa}
A.~Vicente,
%``Theory status and implications of $R_K^{(\ast)}$,''
[arXiv:2001.04788 [hep-ph]].
%1 citations counted in INSPIRE as of 17 Apr 2020

\bibitem{Aaij:2019wad}
  R.~Aaij {\it et al.} [LHCb Collaboration],
 % ``Search for lepton-universality violation in $B^+\to K^+\ell^+\ell^-$ decays,''
  arXiv:1903.09252 [hep-ex].
  %%CITATION = ARXIV:1903.09252;%%
  
  \bibitem{BelleRK}
  S.~Choudhury for the Belle collaboration, ``Measurement of Lepton Flavour Universality in $B$ decays at Belle'', Talk at `EPS-HEP Conference 2019'.

  \bibitem{Aaij:2014pli}
  R.~Aaij {\it et al.} [LHCb Collaboration],
 % ``Differential branching fractions and isospin asymmetries of $B \to K^{(*)} \mu^+ \mu^-$ decays,''
  JHEP {\bf 1406} (2014) 133
 % doi:10.1007/JHEP06(2014)133
  [arXiv:1403.8044 [hep-ex]].
  %%CITATION = doi:10.1007/JHEP06(2014)133;%%


 \bibitem{Abdesselam:2019wac}
  A.~Abdesselam {\it et al.} [Belle Collaboration],
  %``Test of lepton flavor universality in ${B\to K^\ast\ell^+\ell^-}$ decays at Belle,''
  arXiv:1904.02440 [hep-ex].
  %%CITATION = ARXIV:1904.02440;%%

\bibitem{Aaboud:2018mst}
  M.~Aaboud {\it et al.} [ATLAS Collaboration],
 % ``Study of the rare decays of $B^0_s$ and $B^0$ mesons into muon pairs using data collected during 2015 and 2016 with the ATLAS detector,''
  %Submitted to: JHEP
  [arXiv:1812.03017 [hep-ex]].
  %%CITATION = ARXIV:1812.03017;%%

\bibitem{FLAG}
  S.~Aoki {\it et al.} [Flavour Lattice Averaging Group],
 % ``FLAG Review 2019,''
  arXiv:1902.08191 [hep-lat].
  %%CITATION = ARXIV:1902.08191;%%
  
  \bibitem{Wehle:2016yoi}
  S.~Wehle {\it et al.} [Belle Collaboration],
  %``Lepton-Flavor-Dependent Angular Analysis of $B\to K^\ast \ell^+\ell^-$,''
  Phys.\ Rev.\ Lett.\  {\bf 118} (2017) no.11,  111801
%  doi:10.1103/PhysRevLett.118.111801
  [arXiv:1612.05014 [hep-ex]].
  %%CITATION = doi:10.1103/PhysRevLett.118.111801;%%
  %195 citations counted in INSPIRE as of 25 Apr 2019


\bibitem{Abdesselam:2016llu}
   A.~Abdesselam {\it et al.} [Belle Collaboration],
  %``Angular analysis of $B^0 \to K^\ast(892)^0 \ell^+ \ell^-$,''
  arXiv:1604.04042 [hep-ex].
  %%CITATION = ARXIV:1604.04042;%%
  %139 citations counted in INSPIRE as of 25 Apr 2019
  
  \bibitem{Calibbi:2017qbu}
  L.~Calibbi, A.~Crivellin and T.~Li,
 % ``Model of vector leptoquarks in view of the $B$-physics anomalies,''
  Phys.\ Rev.\ D {\bf 98} (2018) no.11,  115002
  % doi:10.1103/PhysRevD.98.115002
  [arXiv:1709.00692 [hep-ph]].
  %%CITATION = doi:10.1103/PhysRevD.98.115002;%%
  
  
  \bibitem{Crivellin:2019dun}
  A.~Crivellin, D.~Muller and C.~Wiegand,
  % ``$b\to s\ell^+\ell^-$ Transitions in Two-Higgs-Doublet Models,''
  arXiv:1903.10440 [hep-ph].
  %%CITATION = ARXIV:1903.10440;%%

\bibitem{Bobeth:2016llm}
  C.~Bobeth, A.~J.~Buras, A.~Celis and M.~Jung,
  %``Patterns of Flavour Violation in Models with Vector-Like Quarks,''
  JHEP {\bf 1704} (2017) 079
  %doi:10.1007/JHEP04(2017)079
  [arXiv:1609.04783 [hep-ph]].
  %%CITATION = doi:10.1007/JHEP04(2017)079;%%

\end{thebibliography}

\end{document}
